\section{Introduction}

In our increasingly digitalized world, the menace of fraudulent academic certificates has persisted for years\cite{forbes2023}, affecting not just ordinary individuals but even public figures at the highest echelons of government \cite{actionsa2023} and doctors wielding forged credentials\cite{forbes2023}. Imagine a scenario where many around the globe claim qualifications they don't possess, duping unsuspecting individuals and institutions who often struggle to discern authenticity in this age of sophisticated counterfeits. In today's age, the art of forgery has evolved to such a degree that even experts find themselves second-guessing. For the average individual, differentiating genuine credentials from fake ones becomes a daunting task, often requiring external validation— a process that can be cumbersome or inaccessible \cite{universityAffairs2023}.

In August 2022, law enforcement in Bogotá, Colombia, dismantled a network of forgers producing fake diplomas and certificates \cite{elTiempo2022}. Additionally, in February 2023, a high-ranking official from Colombia's health ministry was discovered to possess two counterfeit university diplomas, the universities he claimed to have received the certificates from publicly denied any such affiliations \cite{rcn2023}. Meanwhile, in Florida, it was revealed that over 7,600 individuals hold fraudulent nursing degrees \cite{telemundo2023}.

As the world grappled with the COVID-19 pandemic, another shadowy crisis emerged: a surge in forged health certificates, including fabricated negative test results and vaccination cards. Think of the immense challenge this posed to authorities, trying their best to uphold public health measures, only to be thwarted by an undercurrent of deception, complicating their efforts to curb the virus's spread.\cite{threatpost2021}\cite{usaAttourney2023}\cite{washingtonPost2023}\cite{fox13now2023}.

It's in such challenging times that innovations like verifiable credentials (VCs) shine as beacons of hope, offering a potential lifeline against these rampant fraudulent activities. VCs, which are digital proofs of qualifications, achievements, or statuses, can be securely stored and shared due to cryptographic protection. They are grounded in standards that render them interoperable, verifiable, and nearly impossible to counterfeit \cite{newYorkTimes2023}.

To combat the rampant issue of counterfeit credentials and bolster security measures, KBC has collaborated with Howest, the University of Applied Sciences in West Flanders, to introduce Europe's premier digital student card. Conforming to the latest European identity and data standards, this card will be housed within the KBC Mobile app, offering students easy and secure access to academic services and privileges. Starting from the 2023/2024 academic term, all 9,500 students enrolled at Howest will have the digital means to validate their student status, ensuring heightened privacy protection. The technological foundation of this digital student card is a decentralized database, or blockchain. This ensures that only pertinent data is revealed to those verifying the card's authenticity \cite{kbc2023}.

The European Blockchain Services Infrastructure (\href{https://ec.europa.eu/digital-building-blocks/wikis/display/EBSI/Home}{EBSI}) has been spearheading numerous projects involving VC across different domains, such as education and social security. The aim is to create a common standard for easy verification and almost impossible faking of documents. This includes developing EBSI-conformant digital wallets, issuing and verifying credentials, and engaging various stakeholders in these processes, including universities, employers, IT service providers, and national authorities.

In today's rapidly digitalizing era, the challenge of countering sophisticated, AI-fueled forgery of academic and professional credentials in Latin America, particularly in Colombia, has never been more pressing. This paper will delve into the depth of this pervasive problem and introduce verifiable credentials under the Hyperledger Aries technology as a promising and potent solution. By the conclusion, readers will be equipped with a comprehensive guide tailored for Latin American institutions, underscored by a prototype implementation at \href{https://www.univalle.edu.co/}{Univalle}, demonstrating the feasibility and imperative of this innovative approach


\subsection{Problem Statement}

In the digital era, the bedrock of any online interaction or transaction is trust. Yet, cultivating this trust in a vast, impersonal online landscape proves daunting. The prevailing models, which predominantly depend on centralized authorities, are not only susceptible to security breaches and data misuse but also fall short in flexibility, failing to cater to the multifaceted needs of various users.

Enter the Trust Over Internet Protocol (ToIP)—a beacon of hope that promises a decentralized approach to trust, diminishing dependency on centralized entities and granting individuals greater agency over their personal data. Nevertheless, the road to integrating ToIP protocols is riddled with complexities, necessitating a profound comprehension of the intricacies of this technology.

A conspicuous void exists in the form of accessible, detailed guides that demystify ToIP protocol implementation. This dearth of guidance could stymie ToIP's broader acceptance as potential adopters grapple with its practical application nuances.

This project's essence is to bridge this gap, striving to craft a comprehensive guide tailored to the specificities and roles an organization or individual might play within the ToIP ecosystem. A guide of this nature holds the promise of simplifying ToIP integration, enabling users to unlock its full potential and fostering a digital world where ToIP gains widespread traction.


These are the objectives of the project


\subsubsection{Scope}

This guide primarily targets Latin American entities, with a specific focus on Colombia, but its principles can be adapted to broader contexts. The objective is both to provide a hands-on implementation guide and to address the legal and practical nuances associated with Trust Over Internet protocols in the region.


\subsection{Work Structure}

\subsubsection{Introduction}
\begin{enumerate}
    \item Brief about the importance of Trust Over Internet protocols in the digital age.
    \item The problem statement that led to this research.
\end{enumerate}

\subsubsection{State of the art}
\begin{enumerate}
    \item Comprehensive analysis of the present landscape, highlighting recent advancements.
    \item An overview of established best practices within the Trust Over Internet domain.
\end{enumerate}

\subsubsection{Methodology}
\begin{enumerate}
    \item Deep dive into existing literature to establish a foundation for the project.
    \item Details on creating distinct agents for the roles of Issuer and Verifier within the trust triangle.
    \item Steps and rationale behind using a wallet as the Holder.
    \item Explanation of how interactions between the trust triangle components are charted.
    \item Description of how findings from the above steps will be integrated into a comprehensive guide for Trust Over Internet protocols.
    \item A projection of the time required for each phase of the project.
\end{enumerate}

\subsubsection{Expected Results}
\begin{enumerate}
    \item A detailed forecast of the project's outcomes, spotlighting the creation of the comprehensive guide.
    \item The significance of distinct agents for the Issuer and Verifier in the trust triangle.
\end{enumerate}

\subsubsection{Discussion}
\begin{enumerate}
    \item Potential challenges in implementing the guide.
    \item Implications of the research and its findings for the wider industry.
\end{enumerate}

\subsubsection{Bibliography}
\begin{enumerate}
    \item A meticulously curated list of all references and sources consulted during the research.
\end{enumerate}

