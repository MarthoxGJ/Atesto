\section{Methodology}

The methodology for this project is designed to provide a systematic and comprehensive approach to achieving the main and specific objectives. It involves a combination of research, software development, interaction mapping, and guide creation. Each step of the methodology is designed to build upon the previous one, ensuring a cohesive and thorough exploration and implementation of Trust Over Internet protocols.

\subsection{Literature review and research}
This step involves a thorough review of existing literature and research on Trust Over Internet protocols. It could includes academic papers, technical documentation, open source code, development guides, implementation guides and other relevant resources. The goal is to gain a deep understanding of the current state of the field, identify gaps in knowledge, and determine best practices for implementing Trust Over Internet protocols. To fulfil this step, we must:

\begin{itemize}
    \item Identify relevant sources of information.
    \item Review and summarize key findings from each source.
    \item Identify gaps in the current body of knowledge
    \item Determine best practices for implementing Trust Over Internet protocols.
\end{itemize}

\subsection{Development of agents}
This step involves the design and implementation of software agents that can perform the roles of Issuer and Verifier in a Trust Over IP system. This will require a strong understanding of software development and the specific requirements of Trust Over Internet protocols. To fulfil this step, we must:
\begin{itemize}
    \item Define the functional requirements for the Issuer and Verifier agents.
    \item Design the software architecture for each agent.
    \item Implement the agents using appropriate programming languages and technologies.
    \item Test the agents to ensure they function as expected.
\end{itemize}

\subsection{Implementing the wallet as a holder}
This step involves implementing a wallet to act as a Holder that interacts with the Issuer and Verifier agents using interoperable objects. This will involve more software development and integration work. To fulfil this step, we must:

\begin{itemize}
    \item Define the functional requirements for the Holder wallet.
    \item Design the software architecture for the wallet.
    \item Implement the wallet using appropriate programming languages and technologies.
    \item Integrate the wallet with the Issuer and Verifier agents.
    \item Test the wallet and its interactions with the agents.
\end{itemize}

\subsection{Interaction mapping}
This step involves mapping out the extent and method of interaction between each component of the trust triangle. This will involve creating diagrams and documentation that clearly illustrate how the Issuer, Verifier, and Holder interact within the system. To fulfil this step, we must:
\begin{itemize}
    \item Identify interactions between the Issuer, Verifier, and Holder.
    \item Create diagrams that visually represent these interactions.
    \item Document the purpose and outcome of each interaction.
\end{itemize}

\subsection{Guide creation}
This step involves creating a comprehensive guide based on the research and development work. The guide should explain how to implement Trust Over Internet protocols and be tailored to the specific needs and roles an entity may need to assume within the Trust over IP framework. To fulfil this step, we must:
\begin{itemize}
    \item Outline the structure and content of the guide.
    \item Write the guide, incorporating findings from the research and development of the work.
    \item Review and revise the guide to ensure it is clear and comprehensive.
\end{itemize} 

\subsection{Schedule}

\begin{table}[h]
    \centering
        \begin{tabularx}{\textwidth}{|m{7.5cm}|X|X|}
            \hline
            Activity & Start Date & End Date \\
            \hline
            Identify relevant sources of information & 11-09-2023 & 25-09-2023 \\
            \hline
            Review and summarize findings & 18-09-2023 & 02-10-2023 \\
            \hline
            Identify gaps in the knowledge & 18-09-2023 & 02-10-2023 \\
            \hline
            Determine the practices to implement Trust Over Internet protocols & 18-09-2023 & 02-10-2023 \\
            \hline
            Define the functional requirements for the agents & 25-09-2023 & 09-10-2023 \\
            \hline
            Design the software architecture for each agent & 25-09-2023 & 09-10-2023 \\
            \hline
            Design the agents & 09-10-2023 & 27-11-2023 \\
            \hline
            Implement the agents & 11-10-2023 & 17-11-2023 \\
            \hline
            Test the agents & 16-10-2023 & 17-11-2023 \\
            \hline
            Integrate a wallet with the agents & 17-11-2023 & 24-11-2023 \\
            \hline
            Itenfity interactions between the agents & 25-09-2023 & 24-11-2023 \\
            \hline
            Create interaction diagrams & 25-09-2023 & 24-11-2023 \\
            \hline
            Outline the structure and content of the guide & 25-09-2023 & 24-11-2023 \\
            \hline
            Create the guide & 25-09-2023 & 01-12-2023 \\
            \hline
            Test the components & 25-09-2023 & 24-11-2023 \\
            \hline
            Make revisions based on testing results & 25-09-2023 & 01-12-2023 \\
            \hline
        \end{tabularx}
    \caption{Schedule}
  \end{table}

