\section{Theoretical Framework}

The rapid advancement of internet technologies has created an imperative for robust, secure, and user-centric systems to establish trust over internet protocols. This study aims to develop a comprehensive guide for implementing Trust Over Internet Protocols, specifically tailored to the various roles an entity may need to assume within this framework. Our objective is to demystify and facilitate the application of existing Trust Over Internet systems, providing actionable guidance for users to effectively employ these technologies.

Situated at the intersection between blockchain and self-sovereign identity—both of which address the topic of verifiable credentials—this research contributes to two major fields of study. Given the misconceptions surrounding blockchain, primarily due to its association with cryptocurrencies, there is a pressing need to decouple blockchain from this narrow application and explore its broader capabilities. This theoretical framework serves to provide the intellectual scaffolding for such an endeavor, integrating key concepts like Trust Over Internet, Trust Over IP, Self-Sovereign Identity, blockchain, digital wallets, and digital identity.

Understanding the theoretical principles behind these technologies is crucial, not only for academic rigor but also for the practical application of the guide. By grounding this study in a comprehensive theoretical framework, we aim to offer a nuanced understanding that goes beyond the surface-level associations commonly made about blockchain technology. This will enable users, researchers, and stakeholders to approach the practical applications of Trust Over Internet Protocols with a well-rounded perspective, informed by established theories and models.

To fully grasp the intricacies of this research, it is essential to define key terms and concepts that form the basis of our inquiry. This section elucidates terminology such as 'Trust Over the Internet,' 'Trust Over IP,' 'Self-Sovereign Identity,' and 'Blockchain,' among others. These definitions will not only provide clarity but will also establish a shared language for discussing the complex issues surrounding digital trust and identity. By grounding these terms in a well-defined context, we aim to facilitate a more nuanced understanding of the theoretical underpinnings driving this study.

\subsection{Trust Over Internet}

Trust Over the Internet is a board concept that refers to the intricate web of trust relationships established among various entities over the internet. This term encompasses a wide range of technologies, protocols, and methods used to secure and authenticate digital interactions. It also covers the user's subjective perception of trust, which is shaped by factors like reliability, security, and privacy features of online platforms and services. The concept serves as an umbrella term under which various models and approaches, including Trust Over IP, are developed to address specific challenges in online trust.

\subsection{Trust Over IP}

Trust Over IP (ToIP) is an independent project hosted by the Linux Foundation that aims to provide a robust, common standard and complete architecture for internet-scale digital trust. It combines both cryptographic trust and at the machine layer and and human trust at the business, legal and social layers. the ToIP-enabled internet is a digital trust ecosystem of digital trust ecosystems, where the interconnections between each digital trust ecosystem are facilitate through the ToIP stack.

\subsection{Self-Sovereign Identity (SSI)}

Self-Sovereign Identity is a user-centric approach to identity management that allows individuals to own, control and manage their digital identities without relying on centralized authorities. It is a decentralized identity model that enables users to securely store their identity data on their personal devices, such as smartphones, and selectively disclose it to third parties. SSI is based on the concept of verifiable credentials, which are cryptographically signed credentials that can be verified without relying on a centralized authority.

\subsection{Blockchain}

Blockchain is a decentralized ledger technology that fundamentally transforms how data is stored, authenticated, and exchanged. Unlike traditional centralized databases, which are managed by a single entity, a blockchain is maintained by a distributed network of nodes. Each block in the chain contains a list of transactions, secured using cryptographic algorithms, and is immutable once added. This creates a transparent, secure, and tamper-proof record accessible to all parties. In the context of this research blockchain serves as a critical technological foundation for implementing both Trust Over IP and Self-Sovereign Identity. It offers a robust architecture for creating and maintaining trust over Internet, contributing to the study's overarching aim of facilitating secure, interoperable, and user-centric digital trust systems.

\subsection{Verifiable Credentials}

Verifiable Credentials are digital statements made by an issuer about a subject, which can be independently verified by a third party. In essence, they are the digital counterparts of physical credentials, such as passports, driver's licenses, or academic degrees. These credentials are cryptographically secured, often utilizing blockchain technology or other decentralized systems to ensure their authenticity and integrity. Within this research framework, Verifiable Credentials play a significant role in establishing digital trust, particularly in conjunction with Self-Sovereign Identity and Trust Over IP models. They serve as the building blocks for secure, transparent, and user-controlled identity verification processes, thereby contributing to the overarching goal of achieving robust and interoperable digital trust systems.

\subsection{Digital Wallets}

Digital Wallets refer to software applications or hardware devices designed to securely store and manage an individual's digital assets, including but not limited to digital identity, verifiable credentials and even cryptocurrencies. These wallets enable users to control their own data, offering a user-centric approach to digital assets management. They often employ robust encryption methods to ensure the security and privacy of the stored information. Within the context of this study, Digital Wallets are instrumental in facilitating Self-Sovereign Identity and Trust Over IP frameworks. They act as the interface through which users interact with digital trust systems, storing and providing access to verifiable credentials and other essential digital assets.

\subsection{Digital Identity}

Digital Identity refers to the digital representation of an entity, be it an individual, organization or device, in an online environment. It consists of a set of attributes and credentials that authenticate and differentiate one entity from another. These attributes can range from basic information like usernames and passwords to more complex forms of data like biometric scans and verifiable credentials. In the scope of this research, Digital Identity serves as a foundational element linking together the concepts of Trust OVer Internet, Trust Over IP, and Self-Sovereign Identity. It is the core asset that these frameworks aim to protect, manage and verify, thereby playing a pivotal role in establishing and maintaining digital trust.
